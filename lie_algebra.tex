\documentclass[a4paper, 10pt]{article}

\usepackage[margin=1in]{geometry}
\usepackage{amsfonts, amsmath, amssymb, amsthm}
\usepackage[utf8]{inputenc}
\usepackage[english, main=ukrainian]{babel}
\usepackage{pgfplots}
\usepackage{bm}
\usepackage{physics}
\usepackage[unicode]{hyperref}
\usepackage{tikz-cd}
\usepackage{enumitem}
\usepackage{graphicx}
\usepackage{pgfplots}
\usepackage{pdfpages}
\usepackage{caption}
\usepackage{float}

\usepgfplotslibrary{fillbetween}

\usetikzlibrary{spy}
\usetikzlibrary{fit,matrix}

\def\rightproof{$\boxed{\Rightarrow}$ }

\def\leftproof{$\boxed{\Leftarrow}$ }

\newtheoremstyle{theoremdd}
  {\topsep}
  {\topsep}
  {\normalfont}
  {0pt}
  {\bfseries}
  {}
  { }
  {\thmname{#1}\thmnumber{ #2}\textnormal{\thmnote{ \textbf{#3}\\}}}

\theoremstyle{theoremdd}
\newtheorem{theorem}{Theorem}[subsection]
\newtheorem{definition}[theorem]{Definition}
\newtheorem{example}[theorem]{Example}
\newtheorem{proposition}[theorem]{Proposition}
\newtheorem{remark}[theorem]{Remark}
\newtheorem{lemma}[theorem]{Lemma}
\newtheorem{corollary}[theorem]{Corollary}

\newcommand\thref[1]{\textbf{Th.~\ref{#1}}}
\newcommand\defref[1]{\textbf{Def.~\ref{#1}}}
\newcommand\exref[1]{\textbf{Ex.~\ref{#1}}}
\newcommand\prpref[1]{\textbf{Prp.~\ref{#1}}}
\newcommand\rmref[1]{\textbf{Rm.~\ref{#1}}}
\newcommand\lmref[1]{\textbf{Lm.~\ref{#1}}}
\newcommand\crlref[1]{\textbf{Crl.~\ref{#1}}}

\renewcommand{\qedsymbol}{$\blacksquare$}

\DeclareMathOperator{\charac}{char}


\makeatletter
\renewenvironment{proof}[1][Proof.\\]{\par
\pushQED{\hfill \qed}%
\normalfont \topsep6\p@\@plus6\p@\relax
\trivlist
\item\relax
{\bfseries
#1\@addpunct{.}}\hspace\labelsep\ignorespaces
}{%
\popQED\endtrivlist\@endpefalse
}
\makeatother

\title{Алгебра Лі \\ І курс магістратура, 2 семестр}
    	
\begin{document}
\maketitle
\newpage
%\tableofcontents
%\newpage
%Section 1
%\section{Вступ до алгебри Лі}
\subsection{Означення}
\begin{definition}
\textbf{Алгеброю Лі} назвемо векторний простір $L$ над полем $F$ разом з білінійною формою $[\cdot,\cdot] \colon L \times L \to L$, що задовольняє таким умовам:
\begin{align*}
\begin{tabular}{ll}
1) & $\forall x \in L: [x,x] = 0$ \\
2) & $\forall x,y,z \in L: [x,[y,z]] + [y,[z,x]] + [z,[x,y]] = 0$
\end{tabular}
\end{align*}
Остання рівність схожа на \textbf{тотожність Якобі} із аналітичної геометрії.
\end{definition}

\begin{proposition}
$\forall x \in L: [x,x] = 0 \iff \forall x,y \in L: [y,x] = -[x,y]$.\\
За умовою, що $\charac(F) \neq 2$.\\
\textit{Вправа: довести.}
\end{proposition}

\begin{proof}
\rightproof Дано: $[x,x] = 0$ для всіх $x \in L$. Оберемо довільні $x,y \in L$, тоді звідси $[x+y,x+y] = 0$ за умовою. Зокрема за властивістю білінійної форми, $[x,x] + [x,y] + [y,x] + [y,y] = 0$. Таким чином, $[y,x] = -[x,y]$.
\bigskip \\
\leftproof Дано: $[y,x] = -[x,y]$ для всіх $x,y \in L$. Зокрема якщо $y = x$, то звідси $[x,x] = -[x,x]$. Таким чином, $2[x,x] = 0$, але оскільки $\charac(F) \neq 2$, то ми отримаємо $[x,x] = 0$.
\end{proof}

\begin{example}
Розглянемо векторний простір $\mathbb{R}^3$. Тоді векторний добуток, що задається як $[\vec{x},\vec{y}] = (x_2y_3-x_3y_2, x_3y_1 - x_1y_3, x_1y_2-x_2y_1)$, встановлює алгебру Лі. Інколи векторний добуток позначають $\vec{x} \wedge \vec{y}$, алгебру Лі позначають тут $\mathbb{R}^3_{\wedge}$.\\
Зрозуміло, що в цьому випадку $[\vec{x},\vec{x}] = \vec{0}$, за наишм означенням.\\
Із курса аналітичної геометрії, ми доводили так звану формулу "бац мінус цаб". Завдяки неї, там же ми отримали тотожність Якобі, тобто $[\vec{x},[\vec{y},\vec{z}]] = [\vec{y},[\vec{z},\vec{x}]] + [\vec{z},[\vec{x},\vec{y}]] = \vec{0}$.
\end{example}

\begin{example}
Розглянемо множину $\mathfrak{gl}_n(F)$ -- векторний простір всіх матриць $n \times n$, елементи яких над полем $F$, де білінійна форма визначається таким чином: $[A,B] = AB - BA$.\\
Тоді це утворює алгебру Лі. Вона має особливу назву -- \textbf{загальна лінійна алгебра Лі}.\\
$[A,A] = O$ -- це зрозуміло.\\
$[A,[B,C]] + [B,[C,A]] + [C,[A,B]] = [A, BC-CB] + [B, CA-AC] + [C, AB-BA] = \\ = (A(BC-CB) - (BC-CB)A) + (B(CA-AC) - (CA-AC)B) + (C(AB-BA) - (AB-BA)C) = \\
= ABC - ACB - BCA + CBA + BCA - BAC - CAB + ACB + CAB - CBA - ABC + BAC = O$.
\end{example}

\begin{example}
Розглянемо множину $\mathfrak{gl}(V)$ -- векторний простір всіх лінійних відображень $V \to V$, де $V$ -- векторний простір над полем $F$. Білінійну форму визначимо аналогічно: $[U,W] = UW - WU$.\\
Тоді це утворює алгебру Лі (аналогічним чином, що з матрицею). Це теж називають \textbf{загальною лінійною алгеброю Лі}.
\end{example}
\end{document}